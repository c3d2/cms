\documentclass[12pt]{beamer}
%\documentclass[20pt,handout]{beamer}
\usetheme{Darmstadt}
\usepackage{graphicx}
%\usepackage[german]{babel}
\usepackage[T1]{fontenc}
\usepackage[utf8]{inputenc}
\usepackage{tikz}
\setbeamertemplate{footline}[frame number]

\newcommand{\cc}[1]{\includegraphics[height=4mm]{img/#1.png}}
\usepackage{ifthen}
\newcommand{\license}[2][]{\\#2\ifthenelse{\equal{#1}{}}{}{\\\scriptsize\url{#1}}}
\usepackage{textcomp}

\pgfdeclareimage[height=.6cm]{c3d2logo}{./img/c3d2.pdf} 


\pgfdeclarelayer{foreground}
\pgfsetlayers{main,foreground}
\logo{\pgfputat{\pgfxy(-1,0)}{\pgfbox[center,base]{\pgfuseimage{c3d2logo}}}}


\title{NSA, Prism und co - Wie schützt man sich vor Überwachung?}
\author{\small Marius Melzer \& Stephan Thamm\\\large Chaos Computer Club Dresden}
\date{22.01.2014}

\begin{document}
\maketitle

\section{Einleitung}
\subsection{}

\begin{frame}
  \frametitle{Wer sind wir?}
  \begin{figure}
    \includegraphics[height=0.7\textheight]{img/fingerabdruck.jpg}
  \end{figure}
\end{frame}

\begin{frame}
  \frametitle{Wer sind wir?}
  \begin{figure}
    \includegraphics[height=0.7\textheight]{img/trojaner.jpg}
  \end{figure}
\end{frame}

\begin{frame}
    \frametitle{Wer sind wir?}
    \begin{itemize}
      \item<1-> Chaos Computer Club Dresden (\url{http://c3d2.de})
          \note{}
      \item<2-> Datenspuren: Herbst 2014 \url{http://datenspuren.de}
      \item<3-> Podcasts (\url{http://pentamedia.de})
      \item<4-> Chaos macht Schule
    \end{itemize}
\end{frame}

\begin{frame}
    \frametitle{Bundespräsident Gauck zur NSA-Überwachung}
    \begin{center}
      "`Wir wissen z.B., dass es nicht so ist, wie bei der Stasi und dem KGB, dass es dicke Aktenbände gibt, wo unsere Gesprächsinhalte alle aufgeschrieben und schön abgeheftet sind. Das ist es nicht."'
      \end{center}
\end{frame}

\begin{frame}
    \frametitle{Stasi vs. NSA}
    \includegraphics[height=0.7\textheight]{img/akten1.png}
\end{frame}

\begin{frame}
    \frametitle{Stasi vs. NSA}
    \includegraphics[height=0.7\textheight]{img/akten2.png}
\end{frame}

\begin{frame}
    \frametitle{Merkels Handy}
    \includegraphics[height=0.7\textheight]{img/heise-merkel.png}
\end{frame}

\section{Staaten}
\subsection{}

\begin{frame}
  \begin{center}\Large
    Staatliche Überwachung
  \end{center}
\end{frame}

\begin{frame}
    \frametitle{Tempora}
    \includegraphics[height=0.7\textheight]{img/spiegel-tempora.png}
\end{frame}

\begin{frame}
    \frametitle{Verschlüsselung}
    \begin{itemize}
      \item<2-> symetrische Verschlüsselung
      \item<3-> asymetrische Verschlüsselung
      \item<4-> Woher kommt Vertrauen?
    \end{itemize}
\end{frame}

\begin{frame}
    \frametitle{SSL / TLS}
    \begin{itemize}
      \item<2-> eingesetzt im Web, Mail, ...
      \item<3-> hierarchische Struktur
      \item<4-> gespeicherte Liste von vertrauenswürdigen Zertifikaten
    \end{itemize}
\end{frame}

\begin{frame}
    \frametitle{Emailverschlüsselung}
    \begin{itemize}\Large
      \item GPG / PGP
      \item Thunderbird: Gpg4win
      \item Outlook: Gpg4win
      \item Web: Mailvelope (Firefox, Chrome)
    \end{itemize}
\end{frame}

\begin{frame}
    \frametitle{Von Firefox vertraute Zertifikate}
    \begin{center}
      \includegraphics[height=5cm]{img/zertifikate.png}
    \end{center}
\end{frame}

\begin{frame}
  \frametitle{HTTPS Everywhere}
    \begin{center}
      \includegraphics[height=5cm]{img/https-everywhere.png}
    \end{center}
\end{frame}

\begin{frame}
  \frametitle{Vorratsdatenspeicherung (Deutschland)}
    \begin{center}
      \includegraphics[height=5cm]{img/zeit-vds.png}
    \end{center}
\end{frame}

\begin{frame}
  \frametitle{Vorratsdatenspeicherung (USA)}
    \begin{center}
      \includegraphics[height=5cm]{img/netzpolitik-verizon.png}
    \end{center}
\end{frame}

\begin{frame}
  \frametitle{Metadaten}
  \begin{itemize}
    \item<2-> Handynetz
      \begin{itemize}
        \item<3-> Telefonnummern
        \item<4-> Zeitpunkt und Dauer (Telefonate, SMS)
        \item<5-> Funkzelle (Ort)
      \end{itemize}
    \item<6->Internet
      \begin{itemize}
        \item<3-> IP-Adresse
        \item<4-> Alle Verbindungen
        \item<5-> Email: Adressen von Sender und Empfänger, Zugriff
      \end{itemize}
  \end{itemize}
\end{frame}

\begin{frame}
  \frametitle{Metadaten}
    \begin{center}
      \includegraphics[height=5cm]{img/netzpolitik-verizon.png}
    \end{center}
\end{frame}

\begin{frame}
    \frametitle{Metadaten}
    \includegraphics[height=0.7\textheight]{img/maltespitz.png}
\end{frame}

\begin{frame}
    \frametitle{Prism}
    \includegraphics[height=0.7\textheight]{img/prism.jpg}
\end{frame}

\begin{frame}
  \frametitle{Gegenmaßnahme zu Prism}
  \begin{itemize}
    \item<2-> Dezentrale Dienste:
      \begin{itemize}
        \item<3-> Email
        \item<4-> Jabber
        \item<5-> palava.tv
      \end{itemize}
  \end{itemize}
\end{frame}

\begin{frame}
  \frametitle{Zusammenfassung Staaten}
  \begin{itemize}
    \item Staaten wollen Kontrolle, ggf. Daten verkaufen
    \item Gegenmaßnahmen: Verschlüsselung, Tor, dezentrale Dienste
  \end{itemize}
\end{frame}

\section{Unternehmen}
\subsection{}

\begin{frame}
  \begin{center}\Large
    Überwachung durch Unternehmen
  \end{center}
\end{frame}

\begin{frame}
    \frametitle{Geschäftsmodelle I}
    \begin{itemize}
        \item<2-> Karstadt
        \item<3-> Amazon
        \item<4-> Ebay
        \item<5-> Xing
        \item<6-> Facebook
        \item<6-> Google
    \end{itemize}
\end{frame}

\begin{frame}
    \frametitle{Geschäftsmodelle II}
    \uncover{
        \begin{figure}
            \includegraphics[height=0.6\textheight]{img/business_pigs.jpg}
            \license[http://geekandpoke.typepad.com/geekandpoke/2010/12/the-free-model.html]{\cc{by-sa}}
    \end{figure}}
\end{frame}

\begin{frame}
    \frametitle{Geschäftsmodelle III}
    \begin{center} \Large
        Wofür die ganzen Daten?
    \end{center}
\end{frame}

\begin{frame}
    \frametitle{Datensparsamkeit}
    \begin{itemize}
        \item<2-> Viele Daten zusammen ergeben Profile
        \item<3-> Werden die Daten gebraucht?
        \item<4-> Werden echte Daten gebraucht?
            \begin{itemize}
                \item<5-> Gegenmaßnahme: mailinator.com
            \end{itemize}
    \end{itemize}
\end{frame}

\begin{frame}
    \frametitle{Tracking}
    \begin{itemize}
        \item<2-> Cookies
        \item<3-> Like Buttons
        \item<4-> Werbe- und Statistiknetzwerke
        \item<5-> Gegenmaßnahme: disconnect.me
    \end{itemize}
\end{frame}

\section{Mitmenschen}
\subsection{}

\begin{frame}
  \begin{center}\Large
    Überwachung durch andere Menschen
  \end{center}
\end{frame}

\begin{frame}
    \frametitle{Soziale Netzwerke}
    \begin{itemize}
        \item<2-> Das Internet vergisst nicht!
        \item<3-> Was sollte man (nicht) posten?
            \begin{itemize}
                \item<4-> www.weknowwhatyouredoing.com
            \end{itemize}
        \item<5-> Wer soll den Post (nicht) erhalten?
        \item<6-> Beeinträchtigt der Post andere?
        \item<7-> Gegenmaßnahme: Privatsphäre-Einstellungen
        \item<8-> Gegenmaßnahme: Pseudonymität
    \end{itemize}
\end{frame}

\begin{frame}
    \frametitle{Passwörter}
    \begin{itemize}
        \item<2-> Keine einfachen Wörter
        \item<3-> Groß-, Kleinbuchstaben, Ziffern, Sonderzeichen
        \item<4-> Beispiele:
            \begin{itemize}
                \item<5-> dragon
                \item<6-> (nCuAj.§Tsm!f
                \item<7-> IchLiebeDich
                \item<8-> .§)=")=`
                \item<9-> 123456
                \item<10-> qwerty
                \item<11-> Mks?o/.u,ePsw!
            \end{itemize}
        \item<12-> Verschiedene Passwörter nutzen!
    \end{itemize}
\end{frame}

%\ MORE STUFF

\begin{frame}
    \frametitle{Diskussion}
    \begin{center} {\Large Diskussion}\\Marius Melzer und Stephan Thamm\\CMS Dresden: schule@c3d2.de \end{center}
\end{frame}

\end{document}

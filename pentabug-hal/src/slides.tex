\documentclass[12pt]{beamer}
%\documentclass[20pt,handout]{beamer}
\usetheme{Darmstadt}
\usepackage{graphicx}
\usepackage{listings}
\lstset{language=C}
%\usepackage[german]{babel}
\usepackage[T1]{fontenc}
\usepackage[utf8]{inputenc}
\usepackage{tikz}
\setbeamertemplate{footline}[frame number]

\newcommand{\cc}[1]{\includegraphics[height=4mm]{img/#1.png}}
\usepackage{ifthen}
\newcommand{\license}[2][]{\\#2\ifthenelse{\equal{#1}{}}{}{\\\scriptsize\url{#1}}}
\usepackage{textcomp}

\pgfdeclareimage[height=.6cm]{c3d2logo}{./img/c3d2.pdf} 


\pgfdeclarelayer{foreground}
\pgfsetlayers{main,foreground}
\logo{\pgfputat{\pgfxy(-1,0)}{\pgfbox[center,base]{\pgfuseimage{c3d2logo}}}}


\title{Pentabug - Einführung in HAL}
\author{\small Paul Schwanse \\\large Chaos Computer Club Dresden}
\date{12.03.2015}

\begin{document}
\maketitle

\section{Einführung in HAL}

\begin{frame}[fragile]
	\frametitle{Motivation - firmware/apps/example\_1.c}
	\lstinputlisting[language=C,frame=single,basicstyle=\tiny]{/home/koeart/code/pentabug/firmware/apps/example_1.c}
\end{frame}


\begin{frame}
    \frametitle{Motivation}
    \begin{columns}[T]
    \column{.5\linewidth}
    Software
    \begin{itemize}
	    \item Code wiederverwendbar
	    \item einfach benutzbare Funktionen
	    \item wenig Hardwarekenntnisse
    \end{itemize}
    \column{.5\linewidth}
    Hardware
    \begin{itemize}
    \item Anschlüsse können sich ändern
    \item unterschiedliche Hardware, gleiche Funktion
    \item Softwareentwicklung soll einfach sein
    \end{itemize}
    \end{columns}

\end{frame}
\begin{frame}
    \frametitle{Abstraktionsebenen}
    \begin{itemize}
	    \item {Pin Definitionen \textit{<avr/io.h>}}
	    \item {Hardwareabstraktionslayer \textit{firmware/lib/hal.c}}
	    \item {Module \textit{firmware/lib/}}
	    \item {Hauptprogramm (Endlosschleife) \textit{main.c}}
	    \item {Software \textit{firmware/apps/}}
    \end{itemize}
    Einbinden mit
    \begin{itemize}
	    \item AVR Bibliotheken \textit{\#include <avr/io.h>}
	    \item in ``include'' Ordner vorhandene Libs \textit{\#include <pentabug/hal.h>}
	    \item sonstige Libs \textit{\#include ''./module/meinmodul''}
    \end{itemize}
\end{frame}

\begin{frame}
    \frametitle{HAL - Hardware Abstraction Layer}
    \begin{itemize}
	    \item ``Grenze'' zwischen Hardwareansteuerung und Software
	    \item stellt für die benötigte Hardware einfache Funktionen zur Ansteuerung zur Verfügung
    \end{itemize}
    \begin{definition}
	    Details an Hardware können geändert werden, ohne dass Apps geändert werden müssen.
    \end{definition}
     \begin{definition}
	    Programmierer kann für unterschiedliche Hardware gleiche Programme nutzen.
    \end{definition}
\end{frame}

\begin{frame}
    \frametitle{HAL - Pentabug}
    \begin{itemize}
	    \item {Funktionen: \textit{firmware/lib/hal.c}}
	    \item {Benutzung: einbinden von \textit{<pentabug/hal.h>}}
	    \item {Dokumentation: \textit{firmware/doc/Befehle.odt}}
    \end{itemize}
\end{frame}
\begin{frame}[fragile]
	\frametitle{example\_3.c}
	\lstinputlisting[language=C,frame=single,basicstyle=\tiny]{/home/koeart/code/pentabug/firmware/apps/example_3.c}
\end{frame}
\end{document}

\documentclass[table]{beamer}
%\documentclass[14pt,handout]{beamer}
\usetheme{Darmstadt}
\usepackage{graphicx}
\usepackage{hyperref}
\usepackage[german]{babel}
\usepackage[T1]{fontenc}
\usepackage[utf8]{inputenc}
\usepackage{color}
\setbeamertemplate{footline}[frame number]
%\usepackage{enumitem}

\usepackage{pdfcomment}
\newcommand{\ben}[1]{\pdfcomment[author=Ben]{#1}}
\newcommand{\cc}[1]{\includegraphics[height=4mm]{img/#1.png}}
\usepackage{ifthen}
\newcommand{\license}[2][]{\\#2\ifthenelse{\equal{#1}{}}{}{\\\scriptsize\url{#1}}}
\newcommand{\Cc}[1]{\begin{center}
    \includegraphics[height=20mm]{img/200px-Cc-#1.png}
\end{center}}
\usepackage{textcomp}

\usepackage{multirow}
\usepackage{xcolor}


\title{Freie Lizenzen bei Lehr- und Lernmaterialien}
\author{Chaos Computer Club Dresden\\Marius Melzer, Stephan Thamm\\12.03.2013}
\date{Diese Folien gibt es auch öffentlich unter: http://c3d2.de/schule.html}

\begin{document}
\maketitle

\frame{\tableofcontents[hideallsubsections]}

\section{Einleitung}
\subsection{}

\begin{frame}
    \frametitle{Wer sind wir?}
    \begin{itemize}
        \item<2-> Chaos Computer Club Dresden (\url{http://c3d2.de})
            \note{}
        \item<3-> Datenspuren (\url{http://datenspuren.de})
        \item<4-> Podcasts (\url{http://pentamedia.de})
        \item<5-> Chaos macht Schule
            \begin{itemize}
                \item<2-> \url{http://ccc.de/schule}
                \item<2-> \url{http://c3d2.de/schule.html}
            \end{itemize}
        \item<6-> Keine Anwälte
    \end{itemize}
\end{frame}

\section{Freies Wissen}
\subsection{}

\begin{frame}
    \begin{center}\Large
    Freies Wissen
    \end {center}
\end{frame}

\begin{frame}
    \frametitle{Wissen}
    \begin{itemize}
      \item<2-> Einfache Nutzung
      \begin{itemize}
        \item<3-> Verfügbarkeit ohne Zugangsbeschränkungen
        \item<4-> Unabhängig von Zeit und Raum
        \item<5-> Weiterverteilung möglich
      \end{itemize}
      \item<6-> Kollaboratives Erstellen
      \begin{itemize}
        \item<7-> Ermöglicht durch neue Medien
        \item<8-> In gemeinschaftlichem Entwicklungsprozess
        \item<9-> Profitieren von den Änderungen anderer
      \end{itemize}
    \end{itemize}
\end{frame}

\begin{frame}
    \frametitle{Urheberrecht}
    \begin{itemize}
      \item<2-> Urheberrecht
      \item<3-> Public Domain
      \item<4-> Lizenzen
    \end{itemize}
\end{frame}

\begin{frame}
    \frametitle{Freie Software}
    \begin{itemize}
      \item<2-> GNU Projekt
      \begin{itemize}
        \item<3-> 1983 von Richard Stallman gegründet
        \item<4-> GPL = GNU General Public License
        \item<5-> Copyleft
      \end{itemize}
      \item<6-> Beispiele freier Software
      \begin{itemize}
        \item Firefox
        \item OpenOffice.org / LibreOffice
        \item Android
        \item Linux
        \item VLC Media Player
      \end{itemize}
    \end{itemize}
\end{frame}

\begin{frame}
    \frametitle{Freies Wissen}
    \begin{itemize}
      \item<2-> Wikipedia
      \item<3-> Open Street Maps
      \item<4-> ...
      \item<5-> Lehrmaterialien
    \end{itemize}
\end{frame}
 
\section{Creative Commons}
\subsection{}

\begin{frame}
    \begin{center}\Large
    Creative Commons
    \end {center}
\end{frame}

\begin{frame}
    \frametitle{Creative Commons}
    \begin{itemize}
        \item<2-> Organisation hat sich zum Ziel gesetzt, Lizenzen zu erarbeiten
        \item<3-> Eine Lizenz, die ...
            \begin{itemize}
                \item<4-> das Urheberrecht ergänzt (nicht ersetzt)
                \item<5-> eindeutig ist
                \item<6-> leicht verständlich ist
                \item<7-> Nutzung von Werken regelt
            \end{itemize}
        \item<8-> Besteht aus 4 Modulen: BY, SA, ND, NC
    \end{itemize}
\end{frame}

\begin{frame}
    \frametitle{Modul 1: BY (By)}
    \begin{itemize}
        \item Namensnennung des Original Urhebers
        \item implizit ab CC 3.0
        \item Urheberverweis kann auf Wunsch zurückgezogen werden
    \end{itemize}
        \Cc{by}
\end{frame}

\begin{frame}
    \frametitle{Modul 2: SA (ShareAlike)}
    \begin{itemize}
        \item "`Weitergabe unter gleichen Bedingungen"'
        \item Werk darf frei kopiert und weitergegeben werden
        \item Werk darf in Gänze oder Auszugsweise bearbeitet und verwendet werden
        \item Die Lizenz muss unverändert beibehalten werden
    \end{itemize}
    \Cc{sa}
\end{frame}

\begin{frame}
    \frametitle{Modul 3: ND (NonDerivative)}
    \begin{itemize}
        \item "`Keine Bearbeitung"'
        \item Werk darf frei kopiert und weitergegeben werden
        \item Werk darf nicht bearbeitet werden
        \item Verwendung in einer Collage nicht möglich
    \end{itemize}
    \Cc{nd}
\end{frame}

%nd kurz erklärt
\begin{frame}
    \frametitle{Modul 4: NC (NonCommercial)}
    \begin{itemize}
        \item "`Nur nicht-kommerzielle Nutzung"'
        \item Werk darf frei kopiert und weitergegeben werden
        \item Werk darf nicht in kommerziellen Produkten vorkommen
        \item Probleme: Was ist kommerziell?
    \end{itemize}
    \Cc{nc}
\end{frame}

\begin{frame}
    \frametitle{Baukastensystem}
    \begin{itemize}
        \item<1-> Nutzungsbedingungen der Inhalte ist Nutzern klar, da diese die Lizenz erkennen
        \item<2-> Lizensierung des eigenen Werkes einfach möglich
        \item<3-> durch Kombination sind 6 Lizenzen zusammenstellbar
    \end{itemize}
\end{frame}

\begin{frame}
    \frametitle{Weitere freie Lizenzen}
    \begin{itemize}
        \item<2-> Open Database License (ODbL)
        \item<3-> Charityware
        \item<4-> Pizzaware
        \item<5-> Eigene Lizenztexte und Derivate
        \item<6-> Unbekannte Lizenzen sind Mehraufwand für Nutzer
    \end{itemize}
\end{frame}

\section{Inhalte finden}
\subsection{}

\begin{frame}
    \begin{center}\Large
    CC-Inhalte finden
    \end {center}
\end{frame}

\begin{frame}
    \begin{center}\Large
      http://search.creativecommons.org/
    \end {center}
\end{frame}

\begin{frame}
  \frametitle{Wikimedia Commons}
  \includegraphics[width=\textwidth]{img/wikicommons.png}
\end{frame}

\begin{frame}
    \begin{center}\Large
      https://archive.org
    \end {center}
\end{frame}

\begin{frame}
    \begin{center}\Large
      https://wikibooks.org
    \end {center}
\end{frame}

\begin{frame}
  \frametitle{Wikisource}
  \includegraphics[width=\textwidth]{img/wikisource.png}
\end{frame}

\begin{frame}
  \frametitle{Edutags}
  \includegraphics[width=\textwidth]{img/edutags.png}
\end{frame}

\begin{frame}
  \frametitle{Bücher}
  \includegraphics[width=\textwidth]{img/openlibrary.png}
\end{frame}

\begin{frame}
  \frametitle{Bücher}
  \includegraphics[width=\textwidth]{img/gutenberg.png}
\end{frame}

\begin{frame}
  \frametitle{Bücher}
  \includegraphics[width=\textwidth]{img/feedbooks.png}
\end{frame}

\begin{frame}
  \frametitle{Open Educational Resources}
    \begin{itemize}
        \item<2-> Materialien sind wie folgt zugänglich:
            \begin{itemize}
                \item<3-> ohne Einschränkung
                \item<4-> in einem freien Format
                \item<5-> unter einer offenen Lizenzen 
            \end{itemize}
        \item<6-> Begriff umfasst alle Arten von (Lehr-)Materialien
        \item<7-> Idee: Wissensunterschiede zwischen Industrienationen und Entwicklungsländern abbauen
        \item<8-> 2007: Cape Town Open Education Declaration
        \item<9-> Beispiel: Polen lässt Unis freie Lehrmaterialien erstellen
    \end{itemize}
\end{frame}

\begin{frame}
    \begin{center}\Large
    http://timms.uni-tuebingen.de
    \end {center}
\end{frame}

\begin{frame}
    \frametitle{Open Courseware}
      \begin{itemize}
        \item<2-> Berkley: http://webcast.berkeley.edu/
        \item<3-> MIT: http://ocw.mit.edu
        \item<4-> (Coursera: https://www.coursera.org/)
    \end{itemize}
\end{frame}

\begin{frame}
    \frametitle{Weiteres}
      \begin{itemize}
        \item<2-> CC Content Directories: http://wiki.creativecommons.org/Content\_Directories
        \item<3-> Bundesarchiv: http://bundesarchiv.de
      \end{itemize}
\end{frame}

\section{Selbst Lizensieren}
\subsection{}

\begin{frame}
    \begin{center}\Large
    Selbst Lizensieren
    \end {center}
\end{frame}

\begin{frame}
    \frametitle{Lizensierung eigener Werke}
    \begin{itemize}
        \item<2-> Es muss vermerkt werden:
            \begin{enumerate}
                \item<3-> Original Urheber
                \item<4-> Lizenz
                \item<5-> Link zur Lizenz
            \end{enumerate}
        \item<6-> kann nicht zurückgezogen werden
        \item<7-> nicht exklusiv
        \item<8-> Maschinenlesbar durch CC REL
    \end{itemize}
\end{frame}

\begin{frame}
    \begin{center}\Large
    http://creativecommons.org/choose/
    \end {center}
\end{frame}

\section{Fazit}
\subsection{}

\begin{frame}
    \frametitle{Fazit}
    \begin{itemize}
      \item<2-> Lizenzen zum einfachen ``Befreien'' von Inhalten
      \item<3-> Wissen baut auf Wissen auf
      \item<4-> Viele Angebote auf Englisch
      \item<5-> Prosument
    \end{itemize}
\end{frame}

\begin{frame}
  \frametitle{Diskussion}
  \begin{itemize}
    \item Vielen Dank für ihre Aufmerksamkeit
    \item \url{http://c3d2.de/schule.html}
    \item \url{schule@c3d2.de}
    \item Für weitere Informationen (u.a. diese Folien) besuchen Sie bitte \url{http://c3d2.de/schule.html}
  \end{itemize}
  \begin{center}
    \href{https://creativecommons.org/licenses/by-sa/4.0/}{\cc{by-sa}} \\
    \href{https://github.com/c3d2/cms/tree/master/2014_03_12_freie_lizenzen}{\textcolor{blue}{Folien}} vom Chaos Computer Club Dresden
  \end{center}
\end{frame}

\begin{frame}
  \frametitle{Links}
  \begin{itemize}
    \item http://search.creativecommons.org
    \item http://commons.wikimedia.org
    \item http://archive.org
    \item http://wikibooks.org
    \item http://wikisource.org
    \item http://edutags.de
    \item http://openlibrary.org
    \item http://www.gutenberg.de
    \item http://feedbooks.com/publicdomain
    \item http://wiki.creativecommons.org/Content\_Directories
    \item http://bundesarchiv.de
  \end{itemize}
\end{frame}

\end{document}

% Adapted from course material from "Wissenschaftliches Arbeiten mit LaTeX",
% ⓒ 2016 Tom Hanika and Daniel Borchmann, cc-by-sa

\documentclass{scrartcl}

\usepackage[T1]{fontenc}
\usepackage[utf8]{inputenc}
\usepackage[ngerman]{babel}

\usepackage[scale=0.95]{tgpagella}
\usepackage[scale=0.92]{tgheros}
\usepackage[scaled=0.83]{beramono}
\usepackage{mathpazo}
\usepackage{pdfpages}
\usepackage{csquotes}

\usepackage{listings,xcolor}
\lstset{language=[LaTeX]TeX,
  basicstyle=\ttfamily,
  keywordstyle={},
  frame=tb,
  extendedchars=true,
  literate=%
    {ä}{{\"a}}1
    {ö}{{\"o}}1
    {ü}{{\"u}}1
    {Ä}{{\"A}}1
    {Ö}{{\"O}}1
    {Ü}{{\"U}}1
    {ß}{{\ss}}1,
  numbers=none,
  numberstyle=\tiny,
  stepnumber=1,
}

\usepackage{etoolbox}
\BeforeBeginEnvironment{lstlisting}{\medskip}
\AfterEndEnvironment{lstlisting}{\medskip}

\usepackage{tikz}
\usepackage{graphicx}
\usepackage{url}
\usepackage{amsmath}

\begin{document}

\title{Übungen zu Präsentationen mit \LaTeX{}-\texttt{beamer}}
\date{2018-09-08}

\maketitle

Hier eine kleine \textsf{beamer}-Präsentation zum Anfangen.

\begin{lstlisting}
\documentclass[ngerman]{beamer}
\usepackage[utf8]{inputenc}
\usepackage[T1]{fontenc}
\usepackage{babel}

\usetheme{Dresden}
\title[Zeitreisen heute]{Über die Möglichkeit von Zeitreisen}
\author[Doc Brown et al.]{Doc Brown, Danny McFly, Tom McFly}
\date{\today}
\institute[CmS]{Chaos macht Schule}

\begin{document}
  \frame{\titlepage}
  \section{Einleitung}
  \begin{frame}
    \frametitle{Historischer Überblick.}
    Wie wir seit H.\,G.\,Wells wissen \ldots
    \begin{block}{Zusammenfassung}<2->
      Der Flux-Kompensator ist wichtig!
    \end{block}
  \end{frame}

\end{document}
\end{lstlisting}

Nachdem du den Code eingegeben hast, führe die folgenden Schritte aus:
\begin{enumerate}
\item Ändere das genutzte \texttt{beamer}-Thema (\lstinline|usetheme|)
  nacheinander in \texttt{Copenhagen}, \texttt{Darmstadt}, \texttt{Goettingen},
  \texttt{PaloAlto}, \texttt{CambridgeUS}.
\item Ändere alle Daten nach eigenen Ideen, d.\,h.\ Autor, Titel,
  Hochschule, etc.
\item Füge ein weiteres \lstinline|frame| hinzu, welches einen von dir
  gewählten Titel trägt. Dieses \lstinline|frame| soll aus drei
  Blöcken bestehen, mit jeweils einer von dir gewählten
  Blocküberschrift sowie ein paar Wörtern Inhalt.
\item Verändere das eben erstellte \lstinline|frame|, so dass die
  Blöcke nacheinander erscheinen.
\item Füge ein weiteres \lstinline|frame| hinzu, welches den Titel
  \enquote{Achtung} trägt, und eine Liste von fünf Stichpunkten enthält, welche
  nacheinander aufgedeckt werden.

\item Füge folgendes \lstinline|frame| hinzu:
  \begin{lstlisting}
\begin{frame}\frametitle{Cover Story}
  Der folgende Text wird \uncover<2->{erst nach und
   nach} \only<3-4>{nicht} \uncover<1-4>{sinnvoll} \visible<6->{lesbar.}
\end{frame}
   \end{lstlisting}

   Versuche anhand des entstandenen \lstinline|frame| das unterschiedliche
   Verhalten von \lstinline|uncover|, \lstinline|visible|, und \lstinline|only|
   zu verstehen.

\item Lade das Paket \texttt{listings} und füge folgendes \lstinline|frame|
  hinzu:
\begin{lstlisting}
\begin{frame}[fragile]
\begin{lstlisting}[language=C]
  #include <stdio.h>
  main(){
    printf("Hello World");
  }
\end{lstlisting } % Remove this space
\end{frame}
\end{lstlisting}
Du kannst also leicht Quellcode darstellen.
\item Lade das Paket \texttt{tikz} und füge das folgende
  \lstinline|frame| hinzu:
  \begin{lstlisting}
\begin{frame}\frametitle{Spaß mit TikZ}
  \begin{tikzpicture}[overlay,anchor=south west]
    \draw[blue](0,0) circle(2);
    \draw[red](7,3) circle(1);
  \end{tikzpicture}
  Wichtiger Text
\end{frame}
  \end{lstlisting}
Du kannst also in den Frames mit TikZ \enquote{herummalen}. (Details zu TikZ
nicht in diesem Workshop)

\item Entferne die Navigationssymbole.

\end{enumerate}

\end{document}


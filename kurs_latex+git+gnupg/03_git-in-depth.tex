\documentclass{cms-kurs}

\begin{document}

\title{In-Depth: git}
\author{dbo}
\date{2018-09-08}

\maketitle

\begin{frame}
  \frametitle{Übersicht}
  \tableofcontents{}
\end{frame}

\begin{frame}
  \frametitle{Ziele}

  \onslide<+->

  \begin{itemize}
  \item git verstehen und sicher benutzen
  \item Fokus auf Versionkontrolle von \LaTeX{}
  \item Arbeit mit der Kommandozeile
  \item Arbeit mit github (gitlab, …)
  \end{itemize}

\end{frame}

\section{Grundlegende Arbeitsweise}

\begin{frame}[fragile]
  \frametitle{Commits}

  \onslide<+->

  \begin{itemize}
  \item git sammelt zusammengehörige Änderungen in einem \emph{Commit}
  \item Commits verweisen auf Vorgängern (meistens einer, manchmal zwei oder
    mehr; selten keiner)
  \item Commits werden durch ihren SHA-1 Wert eindeutig identifiziert
  \item Commits mit \enquote{Namen} heißen \emph{Branches}
  \end{itemize}

  \begin{center}
    \begin{tikzpicture}
      \path
      node[commit] (root) {}
      node[commit,right=of root] (1) {}
      node[commit,right=of 1] (2) {}
      node[commit,above right=of 2] (3) {}
      node[commit,right=of 3] (4) {}
      node[commit,below right=of 2] (5) {}
      node[commit,label={right:master},below right=of 4] (6) {}
      node[commit,label={right:dev}] (7) at (5 -| 6) {}
      (root) edge[pred] (1)
      (1) edge[pred] (2)
      (2) edge[pred] (3)
      (3) edge[pred] (4)
      (2) edge[pred] (5)
      (4) edge[pred] (6)
      (5) edge[pred] (6)
      (5) edge[pred] (7)
      ;
    \end{tikzpicture}
  \end{center}

\end{frame}

\begin{frame}
  \frametitle{Repositories}

  \onslide<+->

  \begin{itemize}
  \item Sammlung von Commits und weiteren Informationen
  \item Jedes Repository enthält die \emph{gesamte} Versionshistorie
    (dezentraler Ansatz)
  \item Datenstruktur eines Repositories besteht aus Verzeichnisen und Dateien
    (und kann teilweise direkt editiert werden)
  \end{itemize}

\end{frame}

\begin{frame}[fragile]
  \frametitle{Struktur eines Repositories}

  \onslide<+->

\begin{verbatim}
$ ls -F .git
branches/       HEAD    logs/        refs/
COMMIT_EDITMSG  hooks/  MERGE_RR     rr-cache/
config          index   objects/
description     info/   packed-refs
\end{verbatim}

\end{frame}

\begin{frame}
  \frametitle{Index und Working Tree}

  \onslide<+->

  \begin{block}{Working Tree}
    \begin{itemize}
    \item Dateien, an denen aktuell Änderungen vorgenommen werden können
    \item \emph{HEAD} zeigt auf den aktuell ausgecheckten Commit; Dateien des
      Working Tree basieren auf diesem Commit
    \item Ein Repository kann mehr als einen Working Tree haben
      (→ \gitcmd{worktree})
    \end{itemize}
  \end{block}

  \medskip{}

  \begin{block}{Index}
    \begin{itemize}
    \item Ansammlung von Änderungen für den nächsten Commit
    \item Änderungen in einer Datei können unabhängig voneinander in den Index
      aufgenommen werden
    \end{itemize}
  \end{block}
\end{frame}

\begin{frame}[fragile]
  \frametitle{Konfiguration}

  \onslide<+->

  \begin{block}{Haupt-Konfigurationsdateien}
    \begin{itemize}
    \item \verb|/etc/gitconfig| für systemweite Konfiguration
    \item \verb|~/.config/git/config| und \verb|~/.gitconfig| für benutzerweite
      Konfiguration
    \item \verb|.git/config| für lokale Konfiguration im Repository
    \item \verb|.gitignore| für Dateien, die git ignorieren soll
    \end{itemize}

  \end{block}

  \medskip{}

  \begin{block}{Konfiguration}
    \begin{itemize}
    \item \gitcmd{config} und \gitcmd{ignore}
    \item direktes Editieren
    \end{itemize}

    → \verb|man git-config|
  \end{block}
\end{frame}

\begin{frame}
  \frametitle{Konfiguration}

  \onslide<+->

  \begin{Beispiele}
    \begin{itemize}
    \item \gitcmd{config user.name 'John Doe'}
    \item \gitcmd{config user.email 'john.doe@no.wars'}
    \item \gitcmd{ignore dos.exe}
    \end{itemize}
  \end{Beispiele}

  \bigskip{}

  \begin{block}{Empfehlung}
    Keine (größeren) Binärdateien in einem git-Repository speichern!
  \end{block}

\end{frame}

\begin{frame}
  \frametitle{Dokumentation}

  \onslide<+->

  \begin{itemize}
  \item Direkter Aufruf, z.B.{} \gitcmd{add --help}
  \item Manpages: \lstinline{man gittutorial}
  \item Website: \url{https://git-scm.com}
  \item Pro-Git: \url{https://git-scm.com/book/en/v2}
  \item \url{https://rogerdudler.github.io/git-guide/index.de.html}
  \end{itemize}

\end{frame}

\section{Standard Porcelain Commands}

\begin{frame}
  \frametitle{Repositories verwalten}

  \onslide<+->

  \begin{itemize}
  \item Ein neues Repository anlegen:
    \begin{itemize}
    \item \gitcmd{init}
    \item \gitcmd{init --bare}
    \end{itemize}
  \item Ein bestehendes Repository kopieren (\emph{klonen}): \gitcmd{clone}
  \item Verbindungen zu anderen Repositories (\emph{Remotes}) verwalten:
    \gitcmd{remote}
  \end{itemize}

\end{frame}

\begin{frame}
  \frametitle{Verwaltung des Index}

  \onslide<+->

  \begin{itemize}
  \item Gesamtübersicht über den Index: \gitcmd{status}
  \item Änderungen zum Index hinzufügen:
    \begin{itemize}
    \item \gitcmd{add PATHS}
    \item \gitcmd{add --patch PATHS}, \gitcmd{add -p PATHS}
    \item \gitcmd{add -u}
    \end{itemize}
  \item Änderungen aus dem Index entfernen: \gitcmd{reset PATHS}
  \item Änderungen zwischen Index und Working Tree anzeigen: \gitcmd{diff}
  \item Änderungen zwischen Index und letztem Commit anzeigen: \gitcmd{diff
      --cached}
  \end{itemize}

\end{frame}

\begin{frame}[fragile]
  \frametitle{Commits}

  \onslide<+->

  \begin{itemize}
  \item Aus dem aktuellen Inhalt des Index einen Commit erzeugen:
    \gitcmd{commit}
  \item Commit anzeigen: \gitcmd{show}
  \item Graphische Anzeige: \verb|gitk|
  \end{itemize}

  \onslide<+->

  \begin{block}{Benennung von Commits (oder \emph{Revisions})}
    \begin{itemize}
    \item Direkt mit SHA-1:
      \texttt{3491d63610dd8d973caf2d75ecbb9cb3eb00bf6b},\newline kurz:
      \texttt{3491d6}
    \item Branch-Namen, Tags
    \item Zeitliche Referenz: \verb=HEAD@{5 minutes ago}=
    \item Vorgänger: \verb=master~5=
    \item … → \verb=man gitrevisions=
    \end{itemize}
  \end{block}

\end{frame}

\begin{frame}
  \frametitle{Branches und Tags}

  \onslide<+->

  \begin{itemize}
  \item Branches anzeigen, erzeugen, …: \gitcmd{branch}
  \item Branche wechseln: \gitcmd{checkout}
  \item Commits eines Branches anzeigen: \gitcmd{log}
  \item Tags anzeigen, erzeugen, …: \gitcmd{tag}
  \end{itemize}

\end{frame}

\begin{frame}
  \frametitle{Online-Synchronisation (\emph{Merge Workflow})}

  \onslide<+->

  \begin{itemize}
  \item Commits/Branches aus einem anderen Repository holen: \gitcmd{fetch
      REMOTE BRANCH};
  \item Anderen Branch in aktuellen mergen: \gitcmd{merge BRANCH}
  \item Fetch mit anschließendem Merge: \gitcmd{pull}
  \item Alternative zum Merge: \emph{Rebasing} mit \gitcmd{rebase}
  \end{itemize}

\end{frame}

\begin{frame}
  \frametitle{Offline-Synchronisation (\emph{Patch Workflow})}

  \onslide<+->

  \begin{itemize}
  \item Änderungen als Patches ablegen: \gitcmd{format-patch SINCE}
  \item Patches einspielen: \gitcmd{am *.patch}
  \end{itemize}

\end{frame}

\begin{frame}
  \frametitle{Notfallhilfe}

  \onslide<+->

  Manchmal ist das Repository in einem seltsamen Zustand, und nichts klappt mehr
  … was tun?

  \onslide<+->

  \begin{center}
    \Large\textcolor{red!80!black}{\textbf{DON'T PANIC!}}
  \end{center}

  \smallskip{}

  \textbf{Alles, was in einem Commit gespeichert ist, ist sicher!}

  \medskip{}

  \begin{itemize}[<+->]
  \item Erstmal einen Überblick bekommen: \gitcmd{status}
  \item Unerwarteter Merge Conflict: \gitcmd{merge --abort}
  \item \gitcmd{stash} zum Sichern von lokalen Änderungen
  \item \gitcmd{fsck}, um das Repository zu prüfen
  \item \gitcmd{reset HEAD}, \gitcmd{reset BRANCH}
  \item \gitcmd{checkout commit path} zum selektiven Zurücksetzen (VORSICHT:
    löscht lokale Änderungen!)
  \item Allerletze Maßnahme: \gitcmd{reset --hard BRANCH} (VORSICHT: löscht
    alle uncommitteten Änderungen!)
  \end{itemize}

\end{frame}


\section{GitHub}

\begin{frame}
  \frametitle{Übersicht}

  \onslide<+->

  \begin{itemize}
  \item Populärer zentraler Dienst zum Hosting von Repositories
  \item Gehört (mittlerweile) Microsoft
  \item Hostet Clones vieler bekannter Projekte
  \item Auch: Wikis, Webseiten, …
  \end{itemize}

\end{frame}

\begin{frame}
  \frametitle{Forking, Commenting, Pull Requests, …}

  \onslide<+->

  \begin{itemize}
  \item \emph{Forking}: Cloning eines Repositories, mit Zusatzinformation über
    das Ursprungsrepository
  \item Commits in Forks können mit \emph{Pull Requests} in das
    Ursprungsrepository gebracht werden
  \item Umfangreiche Kommentarfunktion, Markdown-Syntax, Unterstützung bei
    Lizenzwahl
  \end{itemize}

  \medskip{}

  → mal praktisch anschauen

\end{frame}

\end{document}
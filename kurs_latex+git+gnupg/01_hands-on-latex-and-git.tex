\documentclass{cms-kurs}

\title{Hands-On: \LaTeX{} und git}
\author{dbo}
\date{2018-09-07}

\begin{document}

\begin{frame}
  \frametitle{Ziele}

  \onslide<+->

  \begin{itemize}
  \item Erste Erfahrungen mit \LaTeX{} sammeln;
  \item Erste Präsentation mit \LaTeX{} erstellen;
  \item \LaTeX{}-Präsentationen mit git verwalten.
  \item Zusammen an einer Präsentation arbeiten.
  \end{itemize}

\end{frame}

\begin{frame}
  \frametitle{Was ist \LaTeX?}

  \onslide<+->

  \begin{itemize}
  \item Markup-Sprache für den Satz vorrangig akademischer Texte;
  \item Makro-Sammlung auf Basis von \TeX{};
  \item Fokus auf korrekter Typographie;
  \item Sehr vielseitig und damit hohes Verwirrungspotential;
  \end{itemize}

\end{frame}

\begin{frame}
  \frametitle{\LaTeX{}-Beispiel}

  \onslide<+->

  $\to$~Beispiel

\end{frame}

\begin{frame}[fragile]
  \frametitle{Installation}

  \onslide<+->

  \begin{block}{Verwirrungspotential}
    \LaTeX{} besteht aus vielen Komponenten (Distributionen, Engines, Editor,
    Viewer, \ldots)
  \end{block}

  \onslide<+->

  \bigskip{}

  \begin{block}{Distributionen}
    \begin{itemize}
    \item \TeX{}Live ($\to$~\url{https://www.tug.org/texlive/})
      \begin{lstlisting}[language=Bash]
        apt install texlive-full
      \end{lstlisting}
    \item Mik\TeX{} ($\to$~\url{https://miktex.org}) für Windows
    \end{itemize}
  \end{block}

\end{frame}

\begin{frame}
  \frametitle{Texteditoren}

  \onslide<+->

  \begin{block}{Wichtig!}
    \LaTeX-Dateien sind reine \emph{Textdateien}!
  \end{block}

  \onslide<+->

  Wir brauchen also einen Texteditor (am besten einen, der auch \LaTeX{}
  versteht) \dots{} \onslide<+-> da gibt es viele!
  \begin{itemize}
  \item TeXstudio (Cross plattform)
  \item TeXmaker  (Cross plattform)
  \item Kile      (nur unter KDE)
  \item vim mit LaTeX-suite
  \item TeXnicCenter (nur unter Windows)
  \item GNU Emacs mit der Erweiterung AUCTeX.
  \end{itemize}

\end{frame}

\begin{frame}
  \frametitle{Versionsverwaltung mit git}

  \onslide<+->

  \begin{block}{Vorteile von \LaTeX{}}
    Da \LaTeX{}-Dateien reiner Text sind, lassen sie sich mit Versionskontrollen
    kombinieren!
  \end{block}

  \onslide<+->

  \begin{block}{Versionskontrolle mit git}
    \begin{itemize}
    \item Versionskontrolle für viele prominente Projekte (Linux-Kernel, Perl,
      Python, Ruby, Rust, git, \ldots);
    \item Für Programmierer gemacht, aber auch darüber hinaus vielseitig
      einsetzbar;
    \item Basiskonzepte: \emph{Repository}, \emph{Commit}, \emph{Branch}, \ldots
    \item Hauptplatform *nix, unter Windows auch \enquote{benutzbar}
    \end{itemize}
  \end{block}

\end{frame}

\begin{frame}
  \frametitle{Erstellung und Initialer Commit}

  \onslide<+->

  \begin{itemize}
  \item<+-> \gitcmd{init} zum Anlegen eines neuen \emph{Repositories};
  \item<+-> \gitcmd{log} zum Anzeigen des aktuellen
  \item<+-> \gitcmd{branch} zur Anzeige des aktuellen Branches
  \end{itemize}

\end{frame}

\begin{frame}
  \frametitle{Eine kleine \LaTeX-Präsentation}

  \onslide<+->

  $\to$~Beispiel: \LaTeX{}-Präsentation in Repository einchecken, Log ansehen, …

\end{frame}

\begin{frame}
  \frametitle{Übersetzung}

  \TODO{Wie übersetze ich \LaTeX-Dokumente?  (Kommandozeile, Editor)}

\end{frame}

\begin{frame}
  \frametitle{Lokale git-Repositories}

  \TODO{Wie bekomme ich Änderungen ins git Repository?  \lstinline{git commit},
    \lstinline{git add}, \lstinline{git log}, \lstinline{git show},
    \lstinline{git diff} (nicht in der Reihenfolge)}

\end{frame}

\begin{frame}
  \frametitle{Eine gemeinsame Präsentation}

  \TODO{Noch ausformulieren}

  \begin{itemize}
  \item Projekt: Jeder macht eine Folie auf seinem eigenen Branch
  \item Dann ins git
  \end{itemize}

\end{frame}

\begin{frame}
  \frametitle{Änderungen holen}

  \onslide<+->

\end{frame}

\begin{frame}
  \frametitle{\LaTeX{} Fundamentals}

  \onslide<+->

  \TODO{Ein paar grundlegende \LaTeX{}-Befehle, damit man eine eigene kleine
    Folie machen kann}

\end{frame}

\begin{frame}
  \frametitle{Remote git-Repositories}

  \onslide<+->

  \TODO{\gitcmd{push}, \gitcmd{fetch}, Pull Requests, \gitcmd{remote}}

\end{frame}

\begin{frame}
  \frametitle{Merging und Konflikte}

  \TODO{Verschiedene Folien mergen; \lstinline{git merge}, \lstinline{git
      mergetool}}

\end{frame}

\end{document}

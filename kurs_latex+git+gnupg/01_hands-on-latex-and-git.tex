\documentclass{cms-kurs}

\title{Hands-On: \LaTeX{} und git}
\author{dbo}
\date{somewhen}

\begin{document}

\begin{frame}
  \frametitle{Ziele}

  \TODO{Ziele ausformulieren}

\end{frame}

\begin{frame}
  \frametitle{Was ist \LaTeX?}

  \TODO{Helikoptereinführung in \LaTeX}

\end{frame}

\begin{frame}
  \frametitle{Installation}

  \TODO{Kurze Erläuterung zur Installation von \LaTeX, sollte aber idealerweise
    schon vorher gemacht werden}

\end{frame}

\begin{frame}
  \frametitle{Texteditoren}

  \TODO{Kurze Liste verfügbarere \LaTeX-Editoren (aus HTW-Kurs übernehmen)}

\end{frame}

\begin{frame}
  \frametitle{Installation von git}

  \TODO{Was der Titel sagt}

\end{frame}

\begin{frame}
  \frametitle{Erstellung von Repositories}

  \TODO{Kurz zeigen, wie git Repositories aussehen und wie man diese anlegt;
    \lstinline{git init}, \lstinline{git log}}

\end{frame}

\begin{frame}
  \frametitle{Eine kleine \LaTeX-Präsentation}

  \TODO{Idee: hier eine kleine Präsentation zeigen, direkt im Editor (sollte
    hier im Repository als Datei liegen)}

\end{frame}

\begin{frame}
  \frametitle{Übersetzung}

  \TODO{Wie übersetze ich \LaTeX-Dokumente?  (Kommandozeile, Editor)}

\end{frame}

\begin{frame}
  \frametitle{Committing}

  \TODO{Wie bekomme ich Änderungen ins git Repository?  \lstinline{git commit},
    \lstinline{git add}, \lstinline{git log}, \lstinline{git show},
    \lstinline{git diff} (nicht in der Reihenfolge)}

\end{frame}

\begin{frame}
  \frametitle{Eine gemeinsame Präsentation}

  \TODO{Noch ausformulieren}

  \begin{itemize}
  \item Projekt: Jeder macht eine Folie auf seinem eigenen Branch
  \item Dann ins git
  \end{itemize}

\end{frame}

\begin{frame}
  \frametitle{Merging und Konflikte}

  \TODO{Verschiedene Folien mergen; \lstinline{git merge}, \lstinline{git
      mergetool}}

\end{frame}

\end{document}

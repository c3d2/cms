\documentclass{cms-kurs}

\title{Hands-On: \LaTeX{} und git}
\author{dbo}
\date{2018-09-07}

\begin{document}

\begin{frame}
  \frametitle{Ziele}

  \onslide<+->

  \begin{itemize}
  \item Erste Erfahrungen mit \LaTeX{} sammeln;
  \item Erste Präsentation mit \LaTeX{} erstellen;
  \item \LaTeX{}-Präsentationen mit git verwalten.
  \item Zusammen an einer Präsentation arbeiten.
  \end{itemize}

\end{frame}

\section{\LaTeX{} First Contact}

\subsection{Eine Einführung am Beispiel}

\begin{frame}
  \frametitle{Was ist \LaTeX?}

  \onslide<+->

  \begin{itemize}
  \item Markup-Sprache für den Satz vorrangig akademischer Texte;
  \item Makro-Sammlung auf Basis von \TeX{};
  \item Eingabeformat Text, Ausgabeformat meist PDF oder PostScript;
  \item Fokus auf korrekter Typographie;
  \item Sehr vielseitig und damit hohes Verwirrungspotential.
  \end{itemize}

\end{frame}

\begin{frame}
  \frametitle{Eine kleine \LaTeX-Präsentation}

  \onslide<+->

  → Beispiel: \LaTeX{}-Präsentation

\end{frame}

\subsection{Das \LaTeX{} Ökosystem}

\begin{frame}[fragile]
  \frametitle{Installation}

  \onslide<+->

  \begin{block}{Verwirrungspotential}
    \LaTeX{} besteht aus vielen Komponenten (Distributionen, Engines, Editor,
    Viewer, \ldots)
  \end{block}

  \onslide<+->

  \bigskip{}

  \begin{block}{Distributionen}
    \begin{itemize}
    \item \TeX{}Live (→ \url{https://www.tug.org/texlive/}) für
      Linux-basierte Systeme und OSX (→ Mac\TeX)
\begin{lstlisting}[language=Bash]
$ apt install texlive-full
\end{lstlisting}
    \item Mik\TeX{} (→ \url{https://miktex.org}) für Windows.
    \end{itemize}
  \end{block}

\end{frame}

\begin{frame}
  \frametitle{Texteditoren}

  \onslide<+->

  \begin{block}{Wichtig!}
    \LaTeX-Dateien sind reine \emph{Textdateien}!
  \end{block}

  \onslide<+->

  Wir brauchen also einen Texteditor (am besten einen, der auch \LaTeX{}
  versteht) \dots{} \onslide<+-> da gibt es viele!
  \begin{itemize}
  \item TeXstudio (Cross plattform)
  \item TeXmaker  (Cross plattform)
  \item Kile      (nur unter KDE)
  \item vim mit LaTeX-suite
  \item TeXnicCenter (nur unter Windows)
  \item GNU Emacs mit der Erweiterung AUCTeX.
  \end{itemize}

\end{frame}

\begin{frame}
  \frametitle{Übersetzung}

  \onslide<+->

  \begin{itemize}
  \item<+-> Editor: entsprechende Knöpfe drücken (ist bei allen anders …);
  \item<+-> Kommandozeile: \lstinline{pdflatex} (eventuell mehrfach), PDF dann
    im Viewer ansehen.
  \end{itemize}

\end{frame}

\subsection{Basic Markup}

\begin{frame}[fragile]
  \frametitle{\LaTeX{} Fundamentals}

  \onslide<+->

  \begin{itemize}
  \item \lstinline{\tiny}, \lstinline{\small}, \lstinline{\large},
    \lstinline{\Large}, \lstinline{\huge}, \lstinline{\Huge} für Größenangaben
  \item \lstinline{\texttt}, \lstinline{\textit}, \lstinline{\textsf},
    \lstinline{\textbf}, \lstinline{\textsc}, \lstinline{\emph}, … für
    Schriftänderungen
  \item \lstinline=\textcolor{red}{rot}=, … für Schriftfarbenänderungen
  \item \lstinline{itemize}, \lstinline{enumerate} für Aufzählungen
  \item \lstinline{column}, \lstinline{columns} für Spalten auf Folien
  \item \lstinline{\includegraphics} für Einbindung von Bildern
  \end{itemize}

  \onslide<+->

  \medskip{}

  → siehe Beispielpräsentation und \LaTeX-Einführung!

\end{frame}

\section{Versionskontrolle mit git}

\subsection{Hands-On: \LaTeX{} verwalten}

\begin{frame}
  \frametitle{Versionsverwaltung mit git}

  \onslide<+->

  \begin{block}{Vorteile von \LaTeX{}}
    Da \LaTeX{}-Dateien reiner Text sind, lassen sie sich mit Versionskontrollen
    kombinieren!
  \end{block}

  \onslide<+->

  \begin{block}{Versionskontrolle mit git}
    \begin{itemize}
    \item Versionskontrolle für viele prominente Projekte (Linux-Kernel, Perl,
      Python, Ruby, Rust, git, \ldots);
    \item Für Programmierer gemacht, aber auch darüber hinaus vielseitig
      einsetzbar;
    \item Basiskonzepte: \emph{Repository}, \emph{Commit}, \emph{Branch}, \ldots
    \item Hauptplatform *nix, unter Windows auch \enquote{benutzbar}
    \end{itemize}
  \end{block}

\end{frame}

\begin{frame}
  \frametitle{Projekt: Eine gemeinsame Präsentation}

  \begin{itemize}
  \item Jeder macht lokal einige Folien;
  \item Einchecken und hochladen in gemeinsames git-Repository;
  \item Dann zu einer Präsentation zusammenführen.
  \end{itemize}

\end{frame}

\subsection{Grundbegriffe}

\begin{frame}
  \frametitle{Erstellung von Repositories}

  \onslide<+->

  \begin{block}{Repository}
    Zusammenfassung aller relevanten Dateien und deren Änderungen.
  \end{block}

  \onslide<+->

  \begin{itemize}
  \item \gitcmd{init} zum Anlegen eines neuen Repositories;
  \item \gitcmd{log} zum Anzeigen des aktuellen
  \item \gitcmd{branch} zur Anzeige des aktuellen Branches
  \item \gitcmd{clone} zum Klonen bereits vorhandener Repositories
  \end{itemize}

\end{frame}

\begin{frame}
  \frametitle{Lokale git-Repositories}

  \onslide<+->

  \begin{block}{Commit}
    Zusammenfassung bestimmter relevanter Änderungen;
    \enquote{Versionszwischenstände}.
  \end{block}

  \begin{block}{Index}
    Zwischenspeicher für Änderungen, die zu einem Commit zusammengefasst werden
    sollen.
  \end{block}

  \onslide<+->

  \begin{itemize}
  \item Aktuelle Änderungen ansehen mit \gitcmd{diff} und \gitcmd{status};
  \item Hinzufügen zum Index: \gitcmd{add presentation.tex};
  \item Einchecken mit \gitcmd{commit};
  \item Sinnvolle Commit-Message!
  \item Mit \gitcmd{log} und \gitcmd{show} prüfen;
  \end{itemize}

\end{frame}

\begin{frame}
  \frametitle{Separate Branches}

  \onslide<+->

  \begin{block}{Branches}
    Abfolge von Änderungen (Commits); \enquote{verschiedene nebenläufige
      Arbeitsversionen}.
  \end{block}

  \onslide<+->

  \begin{itemize}
  \item \gitcmd{branch} zum Anzeigen des aktuellen Branches und zum
    Erstellen neuer Branches;
  \item \gitcmd{checkout} zum Wechseln von Branches;
  \end{itemize}

\end{frame}


\begin{frame}[fragile]
  \frametitle{Remote git-Repositories}

  \onslide<+->

  \begin{block}{Remote}
    Verweis auf ein anderes Repository
  \end{block}

  \onslide<+->

  \begin{itemize}
  \item \gitcmd{remote} zum Anzeigen und Verwalten von Remotes;
  \item \gitcmd{fetch} zum Hohlen von Änderungen; anschließend \gitcmd{merge}
    oder \gitcmd{rebase} (→ git-Einführung)
  \item \gitcmd{push} zum Hochladen von Änderungen (falls Schreibrechte auf dem
    Repository vorhanden sind)

    \medskip

    Pushen von Branches:
\begin{lstlisting}
$ git push remote mybranch
\end{lstlisting}
  \item Alternativ: \emph{Pull Requests} erstellen oder Offline-Workflow
    (→ git-Einführung)
  \end{itemize}

\end{frame}

\begin{frame}
  \frametitle{Merging und Konflikte}

  \onslide<+->

  \begin{block}{Merging}
    Zusammenführen von Änderungen aus mehreren Branches.
  \end{block}

  \onslide<+->

  \begin{itemize}
  \item \gitcmd{merge branch1} fügt Änderungen aus branch1 in aktuellen Branch
    ein;
  \item kann zu Konflikten führen, die manuell aufgelöst werden müssen;
  \item \gitcmd{mergetool} kann dabei helfen.
  \end{itemize}

\end{frame}

\end{document}

\documentclass{cms-kurs}

\title{In-Depth: git}
\author{dbo}
\date{2018-09-08}

\AtBeginSection{%
  \begin{frame}
    \frametitle{Übersicht}
    \tableofcontents[currentsection]{}
  \end{frame}
}

\begin{document}

\begin{frame}
  \frametitle{Übersicht}
  \tableofcontents{}
\end{frame}

\begin{frame}
  \frametitle{Ziele}

  \onslide<+->

  \begin{itemize}
  \item git verstehen und sicher benutzen
  \item Fokus auf Versionkontrolle von \LaTeX{}
  \item Arbeit mit der Kommandozeile
  \item Arbeit mit github (gitlab, …)
  \end{itemize}

\end{frame}

\section{Grundlegende Arbeitsweise}

\begin{frame}
  \frametitle{Commits}

  \onslide<+->

  \TODO{commits als atomare Blöcke eines Repositories, verweisen auf Vorgänger;
    alles andere baut darauf auf}

  \TODO{Praktisch am Beispiel zeigen, wie sich das zeigt}

\end{frame}

\begin{frame}
  \frametitle{Der Index}

  \onslide<+->

  \TODO{describe what this is and what it's for}

\end{frame}

\begin{frame}
  \frametitle{Struktur eines Repositories}

  \onslide<+->

  \TODO{was Titel sagt}

\end{frame}

\begin{frame}
  \frametitle{Dokumentation}

  \onslide<+->

  \begin{itemize}
  \item Website: \url{git-scm.com}
  \item \TODO{more}
  \end{itemize}

\end{frame}

\section{Standard Porcelain Commands}

\begin{frame}
  \frametitle{Repositories verwalten}

  \onslide<+->

  \gitcmd{clone}, \gitcmd{config}, \gitcmd{remote}

\end{frame}

\begin{frame}
  \frametitle{Verwaltung des Index}

  \onslide<+->

  \TODO{Was Titel sagt}

\end{frame}

\begin{frame}
  \frametitle{Commits}

  \onslide<+->

  \TODO{Was Titel sagt}

\end{frame}

\begin{frame}
  \frametitle{Logs}

  \onslide<+->

  \TODO{Was Titel sagt}

\end{frame}

\begin{frame}
  \frametitle{Online-Synchronisation}

  \onslide<+->

  \TODO{\gitcmd{pull}, \gitcmd{fetch}, \gitcmd{pull}, \gitcmd{merge},
    \gitcmd{rebase}}

\end{frame}

\begin{frame}
  \frametitle{Offline-Synchronisation}

  \onslide<+->

  \TODO{\gitcmd{format-patch}, \gitcmd{am}}

\end{frame}

\begin{frame}
  \frametitle{Merging}

  \onslide<+->

  \TODO{Was Titel sagt}

\end{frame}

\section{Konfiguration}

\begin{frame}
  \frametitle{Wo ist was?}

  \onslide<+->

  \TODO{Konfigurationsdatei erläutern, Beispiele zeigen und sowas}

\end{frame}

\section{GitHub}

\begin{frame}
  \frametitle{Cloning, Commenting, und Pull Requests}

  \onslide<+->

  \TODO{Was Titel sagt}

\end{frame}

\end{document}
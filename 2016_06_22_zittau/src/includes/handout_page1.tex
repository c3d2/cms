{\Large \textbf{Wie schütze ich mich vor Überwachung?}}\\
\textbf{Kontakt:} schule@c3d2.de | \textbf{Website:} http://c3d2.de
\vspace{2mm}

\subsection*{Geräte absichern}
\begin{itemize}
  \item Rechte von Software/Apps einschränken:
  \begin{itemize}
    \item PC: Firewall, Handy: Permissions/Berechtigungen für Apps
  \end{itemize}
  \item Open Source Software verwenden
  \begin{itemize}
    \item für den PC: Linux (Ubuntu oder Linux Mint), Firefox, Thunderbird, VLC
    \item für Android: F-Droid (alternativer App-Store)
    \item für iOS: https://github.com/dkhamsing/open-source-ios-apps
  \end{itemize}
\end{itemize}
  
\subsection*{Verschlüsselung}
\begin{itemize}
  \item ``Transportweg''-Verschlüsselung (= Schloss im Browser)
  \begin{itemize}
    \item Plugin ``HTTPS Everywhere'': damit man das immer nutzt wenn es möglich ist
  \end{itemize}
  \item Ende-zu-Ende-Verschlüsselung
  \begin{itemize}
    \item für direkte Kommunikation besser, weil von Gerät zu Gerät geschützt, auch der Anbieter kann nicht mitlesen
    \item Open Source Apps, die das können: Signal, Jabber (Android: Conversations, iOS: ChatSecure)
  \end{itemize}
\end{itemize}

\subsection*{Metadaten}
\begin{itemize}
  \item Tracking loswerden:
  \begin{itemize}
    \item PC: Plugin ``Privacy Badger'' oder ``Disconnect''
    \item Android: Google-Einstellungen -> Anzeigen -> Interessensbezogene Werbung deaktivieren
    \item iOS: Einstellungen -> Datenschutz -> Werbung -> Kein Ad-Tracking
  \end{itemize}
  \item Alternative Dienste nutzen:
  \begin{itemize}
    \item Suchmaschine: https://startpage.com
    \item Kartendienst: https://osm.org
    \item statt Dropbox und Handy-Sync.: Nextcloud (untersch. Anbieter)
  \end{itemize}
\end{itemize}
